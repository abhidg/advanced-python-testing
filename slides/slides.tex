\documentclass[11pt,xcolor={dvipsnames},hyperref={pdftex,pdfpagemode=UseNone,hidelinks,pdfdisplaydoctitle=true},usepdftitle=false]{beamer}
\usepackage{d1}
\usepackage{minted}
\usemintedstyle{borland}
\setminted{fontsize=\footnotesize}
\hypersetup{pdftitle={Advanced Python Testing}}
\newcommand{\footer}[1]{\parbox{\linewidth}{\vspace*{-8pt}#1\hfill\insertpagenumber}}

% \setbeamertemplate{footline}[text line]{%
%   \parbox{\linewidth}{\vspace*{-8pt}https://github.com/abhidg/advanced-python-testing\hfill\insertpagenumber}}
\setbeamertemplate{footline}[text line]{\footer{https://github.com/abhidg/advanced-python-testing}}
\begin{document}
\title{Advanced Python Testing}


\begin{frame}

  \begin{columns}
  \begin{column}{0.4\textwidth}
    {\LARGE Advanced Python Testing}
    \vskip 0.5cm
    Abhishek Dasgupta
    \vskip 2.5cm
    RSECon24, Newcastle

    2024-09-03
  \end{column}
  \begin{column}{0.4\textwidth}  %%<--- here
    Tired of waiting for network requests, writing millions of edge cases
and copy pasting outputs for testing? This walkthrough is for you.
\vskip 0.5cm
Topics: mocking, property based testing, snapshot testing
  \end{column}
  \end{columns}
  \end{frame}

% \begin{frame}{Advanced Python Testing}
% \phantomsection\label{advanced-python-testing}

% Examples are in Python, but there are similar libraries available in
% other languages.

% \end{frame}

\begin{frame}{Mocking}
\phantomsection\label{mocking}
\begin{itemize}
\item
  Test a code interface without requiring the underlying code to be run
\item
  Useful when testing the actual interface is

  \begin{itemize}
  \item
    time-consuming
  \item
    requires network access
  \item
    not possible in an automated way
  \end{itemize}
\item Pros: reproducibility, easier setup. Cons: hidden behaviour not exposed
\item Alternatives: use docker to setup infrastructure
\end{itemize}
\end{frame}


{
\setbeamertemplate{footline}[text line]{%
\parbox{\linewidth}{\vspace*{-8pt}Example from https://docs.python.org/3/library/unittest.mock.html\hfill\insertpagenumber}}

\begin{frame}[fragile]{Mocking using \texttt{unittest.mock}}
\begin{itemize}
  \item In-built support for mocking in standard library
  \item Mock return values
  \item Assert function called with certain values
  \item Raise exceptions through side effects
\end{itemize}
\vskip 0.5cm
\begin{minted}{python}
    from unittest.mock import MagicMock
    thing = ProductionClass()
    thing.method = MagicMock(return_value=3)
    thing.method(3, 4, 5, key='value')
    thing.method.assert_called_with(3, 4, 5, key='value')
\end{minted}
\end{frame}
}
{
\setbeamertemplate{footline}[text line]{\footer{https://xkcd.com/327}}
\begin{frame}{Property based testing}
  \begin{itemize}
    \item
  Checks that a particular property is satisfied by the code
  \item
  Similar to fuzzing, which tests a system with random inputs
  \end{itemize}
  \vskip 0.7cm
  \includegraphics[width=\textwidth]{xkcd-327.png}
\end{frame}
}
\begin{frame}[fragile]{Property based testing}
\phantomsection\label{property-based-testing}
\begin{itemize}
\item
  Normally one writes example based tests
\end{itemize}

\begin{minted}{python}
    def fahrenheit_to_celsius(temp):
      return 5 * (temp - 32) / 9
    @pytest.mark.parametrize("f,c", [(212, 100), (32, 0)])
    def test_fahrenheit_to_celsius(temp):
      assert fahrenheit_to_celsius(f) == c
\end{minted}

\begin{itemize}
\item
  Cons of example based testing is not being exhaustive enough
\item
  Often we can miss edge cases
\item
  Applications: test input sanitisation, encode/decode round trips
\item
  What is \texttt{fahrenheit\_to\_celsius(NaN)}?
\end{itemize}
\end{frame}

\begin{frame}[fragile]{Property based testing using \texttt{hypothesis}}
\phantomsection\label{property-based-testing-using-hypothesis}
\begin{itemize}
\item
  Test properties without providing examples
\item
  Generate random examples and edge cases to test your code
\item
  Number of examples and search strategies can be customised
\item
  Find counterexamples
\end{itemize}
\vskip 0.5cm
\begin{minted}{python}
    from hypothesis import given
    from hypothesis.strategies import floats, integers
    def divide(x, y): return x / y
    @given(integers(), integers())
    def test_divide(x, y): assert divide(x, y) < x
\end{minted}

\end{frame}


\begin{frame}[fragile]{Snapshot based testing}
\phantomsection\label{snapshot-based-testing}
\begin{itemize}
\item
  Snapshots output of a function or process
\item
  Useful for regression testing, data transformations
\item
  Can be emulated manually (save output to disk, read and test that it
  is the same), but snapshot testing libraries make this easier
\item We will use the \texttt{syrupy} library for this
\end{itemize}
\vskip 0.3cm
\begin{minted}{python}
    def test_something(snapshot):
      data = load_data(something)
      assert function(data) == snapshot
\end{minted}
\end{frame}


\begin{frame}{Walkthrough: finding the sun!}
\phantomsection\label{walkthrough-finding-the-sun}
\begin{itemize}
\item
  \alert{Problem}: Find a block of $N$ hours when it will be sunny and
  clear, given a location.
\item
  We will try to find a block of sun of two hours in Newcastle over the
  next week
\item
  OpenMeteo has a nice free API
\item
  Steps: fetch data, process and find the sun block
\end{itemize}
\end{frame}

\begin{frame}{Further reading}
\begin{itemize}
  \item Mocking: \url{https://docs.python.org/3/library/unittest.mock-examples.html}
  \item Mocking: specialised libraries \texttt{moto} (AWS), \texttt{requests-mock}
  \item Mocks can automatically match specs (\texttt{autospec})
  \item Property based testing \url{https://hypothesis.works/articles/getting-started-with-hypothesis/}
  \item Property based testing \texttt{schemathesis} (API testing), \texttt{hypothesis-jsonschema}
\end{itemize}
\end{frame}
\end{document}
